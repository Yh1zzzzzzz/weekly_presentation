\documentclass[aspectratio=169, 10pt]{beamer} 
% aspectratio=169: 16:9 宽屏
% 10pt: 字体大小

% ==================================================
% 1. 中文支持与基础宏包
% ==================================================
\usepackage[UTF8]{ctex}     % 核心:支持中文
\usepackage{graphicx}       % 图片
\usepackage{booktabs}       % 表格
\usepackage{amsmath,amssymb}% 数学公式
\usepackage{subcaption}     % 子图
\usepackage{listings}       % 代码块
\usepackage{xcolor}         % 颜色
\usepackage{tikz}           % 绘图

% ==================================================
% 2. 主题与导航栏设置 (关键修改)
% ==================================================
% 基础主题:Madrid (底部有作者、日期、页码)
\usetheme{Madrid}           
% 配色方案:Beaver (灰红色系,适合学术)
\usecolortheme{beaver}      

% 【核心修改】顶部导航栏设置
% miniframes: 在顶部显示目录结构和小圆点进度
% subsection=false: 不显示子小节,只显示大章节,保持顶部简洁
\useoutertheme[subsection=false]{miniframes} 

% 稍微调整顶部颜色,使其更清晰
\setbeamercolor{section in head/foot}{fg=white, bg=darkgray}

% ==================================================
% 3. 代码块样式设置 (Python风格)
% ==================================================
\definecolor{codegreen}{rgb}{0,0.6,0}
\definecolor{codegray}{rgb}{0.5,0.5,0.5}
\definecolor{codepurple}{rgb}{0.58,0,0.82}
\definecolor{backcolour}{rgb}{0.95,0.95,0.92}

\lstdefinestyle{pystyle}{
    backgroundcolor=\color{backcolour},   
    commentstyle=\color{codegreen},
    keywordstyle=\color{magenta},
    numberstyle=\tiny\color{codegray},
    stringstyle=\color{codepurple},
    basicstyle=\ttfamily\footnotesize, % 如果代码中文乱码,尝试改成 \small
    breakatwhitespace=false,         
    breaklines=true,                 
    captionpos=b,                    
    keepspaces=true,                 
    numbers=left,                    
    numbersep=5pt,                  
    showspaces=false,                
    showstringspaces=false,
    showtabs=false,                  
    tabsize=2,
    language=Python,
    escapeinside=`` % 允许在代码块里写中文注释(如需)
}
\lstset{style=pystyle}

% ==================================================
% 4. 自定义数学符号
% ==================================================
\newcommand{\vx}{\mathbf{x}}         % 图像向量
\newcommand{\vt}{\mathbf{t}}         % 文本向量
\newcommand{\Lcal}{\mathcal{L}}      % Loss

% ==================================================
% 5. 封面信息
% ==================================================
\title[每周汇报 Week 5]{每周科研进度汇报} % [底部显示的短标题]{封面长标题}
\subtitle{方向:多模态大模型与视觉对齐} 
\author[你的名字]{张三 (Your Name)} 
\institute[xx实验室]{人工智能与多模态实验室}
\date{\today}

% ==================================================
% 正文开始
% ==================================================
\begin{document}

% --- 封面 ---
\begin{frame}
    \titlepage
\end{frame}

% --- 目录页 (可选,因为顶部已经有导航了,这页可以删掉) ---
\begin{frame}{本次汇报目录}
    \tableofcontents
\end{frame}

% ==================================================
% Section 1: 核心摘要
% ==================================================
\section{核心摘要} % 这个标题会显示在顶部导航栏
\begin{frame}{核心摘要 (Executive Summary)}
    \begin{itemize}
        \setlength\itemsep{1em}
        \item \textbf{本周重点 (Key Achievements):} 
            \begin{itemize}
                \item 复现了 \textit{LLaVA} 的多模态对齐模块。
                \item 调试通了 Flickr30k 数据集的预处理代码。
            \end{itemize}
        
        \item \textbf{关键数据 (Highlight):} 
            \begin{itemize}
                \item 在小样本测试中,图文匹配准确率提升了 \textcolor{red}{2.1\%}。
                \item 训练显存占用优化,从 24GB 降至 16GB。
            \end{itemize}
            
        \item \textbf{当前状态:} \textcolor{green!60!black}{\textbf{正常推进 (On Track)}} 
    \end{itemize}
\end{frame}

% ==================================================
% Section 2: 文献阅读
% ==================================================
\section{文献阅读}
\begin{frame}{文献阅读: [论文标题]}
    \framesubtitle{发表于: CVPR 2024 | 机构: OpenAI/Google}

    \begin{columns}[T]
        % 左栏文字
        \begin{column}{0.5\textwidth}
            \textbf{研究动机 (Motivation):}
            \begin{itemize}
                \item 现有的 VLM 在处理细粒度空间关系时表现不佳。
                \item 提出了一种新的 \textbf{Region-Aware Attention} 机制。
            \end{itemize}
            
            \vspace{0.5cm}
            \textbf{核心方法 (Method):}
            \begin{itemize}
                \item 引入 Bounding Box 作为 Prompt。
                \item 损失函数设计:
                $$ \Lcal = \Lcal_{ce} + \lambda \Lcal_{iou} $$
            \end{itemize}
        \end{column}
        
        % 右栏图片
        \begin{column}{0.5\textwidth}
            \begin{figure}
                \centering
                % 请替换为你的图片: \includegraphics[width=\textwidth]{model.png} 
                \rule{\textwidth}{4cm} % 黑色占位符
                \caption{论文提出的模型架构图}
            \end{figure}
        \end{column}
    \end{columns}
\end{frame}

% ==================================================
% Section 3: 方法改进
% ==================================================
\section{方法与改进}
\begin{frame}[fragile]{方法改进:跨模态注意力机制}
    % [fragile] 是代码块必须的选项
    
    \textbf{1. 数学定义:}
    为了增强图像区域 $\vx$ 和文本 Token $\vt$ 的交互,计算如下相似度:
    
    \begin{equation}
        Attention(Q, K, V) = \softmax\left(\frac{Q K^T}{\sqrt{d_k}}\right) V
    \end{equation}
    
    \textbf{2. PyTorch 实现片段:}
    \begin{lstlisting}
# 定义 Cross-Attention 层
class CrossAttention(nn.Module):
    def forward(self, img_feat, text_feat):
        # img_feat: [Batch, Img_Seq, Dim]
        # text_feat: [Batch, Txt_Seq, Dim]
        
        q = self.query(text_feat)
        k = self.key(img_feat)
        
        attn_output, _ = self.mha(q, k, k)
        return attn_output
    \end{lstlisting}
\end{frame}

% ==================================================
% Section 4: 实验结果
% ==================================================
\section{实验结果}
\begin{frame}{实验结果:定性分析 (Qualitative)}
    \textbf{测试 Prompt:} \textit{“一只戴着墨镜的赛博朋克风格猫咪”}
    
    \vspace{0.3cm}
    
    \begin{figure}
        \centering
        % 第一张图
        \begin{subfigure}{0.3\textwidth}
            \centering
            \rule{\linewidth}{2.5cm} % 替换为图片
            \caption{Baseline}
        \end{subfigure}%
        \hspace{0.05\textwidth}
        % 第二张图
        \begin{subfigure}{0.3\textwidth}
            \centering
            \rule{\linewidth}{2.5cm} % 替换为图片
            \caption{Ours (改进版)}
        \end{subfigure}%
        \hspace{0.05\textwidth}
        % 第三张图
        \begin{subfigure}{0.3\textwidth}
            \centering
            \rule{\linewidth}{2.5cm} % 替换为图片
            \caption{Ground Truth}
        \end{subfigure}
    \end{figure}
    
    \textbf{分析:} 可以看到我们的方法在光影细节(墨镜反射)上比 Baseline 更真实。
\end{frame}

% ==================================================
% Section 5: 问题与计划
% ==================================================
\section{问题与计划}
\begin{frame}{当前问题与下周计划}
    \begin{columns}[T]
        \begin{column}{0.48\textwidth}
            \setbeamercolor{block title}{bg=red!80!black,fg=white}
            \begin{block}{遇到的困难 (Issues)}
                \begin{enumerate}
                    \item \textbf{OOM 报错:} 当 Batch Size 大于 32 时显存溢出。
                    \item \textbf{数据脏乱:} 发现 LAION 数据集中有部分图片链接失效,导致 DataLoader 卡死。
                \end{enumerate}
            \end{block}
        \end{column}
        
        \begin{column}{0.48\textwidth}
            \setbeamercolor{block title}{bg=blue!80!black,fg=white}
            \begin{block}{下周计划 (Next Steps)}
                \begin{enumerate}
                    \item 尝试使用 \texttt{Gradient Checkpointing} 技术节省显存。
                    \item 增加异常捕获代码,跳过失效图片。
                    \item 跑完 COCO 数据集的对比实验。
                \end{enumerate}
            \end{block}
        \end{column}
    \end{columns}
\end{frame}

% --- 结束页 ---
\begin{frame}
    \centering
    \Huge \textbf{Q \& A}
    
    \vspace{1cm}
    \large 感谢聆听,请老师指导!
\end{frame}

\end{document}