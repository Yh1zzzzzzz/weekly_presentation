\documentclass[aspectratio=169, 10pt]{beamer} 
% aspectratio=169: 16:9 宽屏
% 10pt: 字体大小

% ==================================================
% 1. 中文支持与基础宏包
% ==================================================
\usepackage[UTF8]{ctex}     % 核心:支持中文
\usepackage{graphicx}       % 图片
\usepackage{booktabs}       % 表格
\usepackage{amsmath,amssymb}% 数学公式
\usepackage{subcaption}     % 子图
\usepackage{listings}       % 代码块
\usepackage{xcolor}         % 颜色
\usepackage{tikz}           % 绘图
\newcommand{\autoslidegraphic}[1]{%
    \includegraphics[width=\linewidth,height=0.7\textheight,keepaspectratio]{#1}}

% ==================================================
% 2. 主题与导航栏设置 (关键修改)
% ==================================================
% 基础主题:Madrid (底部有作者、日期、页码)
\usetheme{Madrid}           
% 配色方案:Beaver (灰红色系,适合学术)
\usecolortheme{beaver}    
\setbeamercolor{structure}{fg=black}           % 所有结构标题(含 frame)用黑色
\setbeamercolor{frametitle}{fg=black} % 如需额外指定背景,可一起写  

% 【核心修改】顶部导航栏设置
% miniframes: 在顶部显示目录结构和小圆点进度
% subsection=false: 不显示子小节,只显示大章节,保持顶部简洁
\useoutertheme[subsection=false]{miniframes} 

% 稍微调整顶部颜色,使其更清晰
\setbeamercolor{section in head/foot}{fg=white, bg=darkgray}

% ==================================================
% 3. 代码块样式设置 (Python风格)
% ==================================================
\definecolor{codegreen}{rgb}{0,0.6,0}
\definecolor{codegray}{rgb}{0.5,0.5,0.5}
\definecolor{codepurple}{rgb}{0.58,0,0.82}
\definecolor{backcolour}{rgb}{0.95,0.95,0.92}

\lstdefinestyle{pystyle}{
    backgroundcolor=\color{backcolour},   
    commentstyle=\color{codegreen},
    keywordstyle=\color{magenta},
    numberstyle=\tiny\color{codegray},
    stringstyle=\color{codepurple},
    basicstyle=\ttfamily\footnotesize, % 如果代码中文乱码,尝试改成 \small
    breakatwhitespace=false,         
    breaklines=true,                 
    captionpos=b,                    
    keepspaces=true,                 
    numbers=left,                    
    numbersep=5pt,                  
    showspaces=false,                
    showstringspaces=false,
    showtabs=false,                  
    tabsize=2,
    language=Python,
    escapeinside=`` % 允许在代码块里写中文注释(如需)
}
\lstset{style=pystyle}

% ==================================================
% 4. 自定义数学符号
% ==================================================
\newcommand{\vx}{\mathbf{x}}         % 图像向量
\newcommand{\vt}{\mathbf{t}}         % 文本向量
\newcommand{\Lcal}{\mathcal{L}}      % Loss

% ==================================================
% 5. 封面信息
% ==================================================
\title[每周汇报]{每周进度汇报} % [底部显示的短标题]{封面长标题}
% \subtitle{方向:多模态大模型与视觉对齐} 
\author[杨淏天]{杨淏天} 
% \institute[xx实验室]{实验室}
\date{\today}

% ==================================================
% 正文开始
% ==================================================
\begin{document}

% --- 封面 ---
\begin{frame}
    \titlepage
\end{frame}

\begin{frame}{本次汇报目录}
    \tableofcontents
\end{frame}

% ==================================================
% Section 1: 核心摘要
% ==================================================
\section{核心摘要} % 这个标题会显示在顶部导航栏
\begin{frame}{核心摘要 (Executive Summary)}
    \begin{itemize}
        \setlength\itemsep{1em}
        \item \textbf{本周重点 (Key Achievements):} 
            \begin{itemize}
                \item 3篇模型微调的论文,均出自一个作者
                \item IRRA:2023年的SOTA
                \item UFineBench:一个新的数据集,同时也是第一个使用LLM来做描述增强的数据集
            \end{itemize}
            
    \end{itemize}
\end{frame}

% ==================================================
% Section 2: 文献阅读
% ==================================================
\section{CLIP微调}
\begin{frame}{CSKT:CLIP-based Synergistic Knowledge Transfer for Text-based Person Retrieval}
    \framesubtitle{发表于: ICASSP 2024}

    \begin{columns}[T]
        % 左栏文字
        \begin{column}{0.5\textwidth}
            \textbf{研究动机 (Motivation):}
            \begin{itemize}
                \item 第一篇对CLIP在行人检索任务上进行微调的论文。
            \end{itemize}
            
            \vspace{0.5cm}
            \textbf{核心方法 (Method):}
            \begin{itemize}
                \item 1.Adapter对encoder的MLP进行微调。
                \item 2.Prompt Tuning对transformer各层的Prompt做调整
            \end{itemize}
        \end{column}
        
        % 右栏图片
        \begin{column}{0.5\textwidth}
            \begin{figure}
                \centering
                \autoslidegraphic{figures/CSKT.png}
                \caption{论文提出的模型架构图}
            \end{figure}
        \end{column}
    \end{columns}
\end{frame}

\begin{frame}{具体的微调算法}
    \begin{columns}[T]
        % 左栏文字
        \begin{column}{0.5\textwidth}
            \textbf{MLP微调:}
            % 调整图片大小,防止溢出下方公式
            \begin{figure}
                \centering
                \includegraphics[width=0.40\linewidth]{figures/adapter.png}
                \caption{MLP-Adapter的结构图}
            \end{figure}
            \begin{itemize}
                \item 对attention后的MLP做微调
            \end{itemize}
            % 将公式移到图片下方,保证显示完整
            \vspace{0.5em}
            \[
                x_{out} = x + x_{MLP} + s \cdot \mathrm{ReLU} \left( \mathrm{LN} (x) \cdot W_d \right) \cdot W_u
            \]
        \end{column}

        % 右栏图片
        \begin{column}{0.5\textwidth}
            \textbf{Prompt Tuning:}
            \begin{itemize}
                \item 对transformer各层的Prompt做调整
                % 使用 \[ ... \] 比 $$...$$ 安全,无歧义
                \[
                  [P_0^t,\; P_0^{v\to t},\; T_0]
                \]
                \item $P_0^t$是初始的文本token,对于下一层来说,是前一层的输出。
                \item $P_0^{v\to t}$是初始的视觉token经过projection后的,
                \item $T_0$是可学习的文本token。
                \item 上述三个向量拼接得倒最终送入当前layer做注意力计算的输入
                % \item 视觉encoder亦然,只是把 $P_0^{v\to t}$替换为 $T_0^{t\to v}$。
            \end{itemize}
        \end{column}
    \end{columns}
\end{frame}

\begin{frame}{DM-Adapter: Domain-Aware Mixture-of-Adapters for Text-Based Person Retrieval}
    \framesubtitle{发表于: AAAI 2025}

    \begin{columns}[T]
        % 左栏文字
        \begin{column}{0.5\textwidth}
            \textbf{研究动机 (Motivation):}
            \begin{itemize}
                \item 引入LLM中的MOE结构,在推理时选择最合适的两个Adapter进行激活
            \end{itemize}
            
            \vspace{0.5cm}
            \textbf{核心方法 (Method):}
            \begin{itemize}
                \item 引入MOE架构,实现Adapter的动态选择。
                \item 引入Load-Balance Loss,确保每个Adapter都得到充分训练。
            \end{itemize}
        \end{column}
        
        % 右栏图片
        \begin{column}{0.5\textwidth}
            \begin{figure}
                \centering
                \autoslidegraphic{figures/DM-adpter.png}
                \caption{论文提出的模型架构图}
            \end{figure}
        \end{column}
    \end{columns}
\end{frame}

\begin{frame}{具体的MOE算法}
    \begin{itemize}
        \item \textbf{MOE激活:}
    \end{itemize}
    \[
        y = h_o + \sum_{i=0}^{n-1} \mathrm{Softmax}\left( \mathrm{TopK}\left(x W + p W_d\right)_i \right) \cdot \mathrm{Adapter}_i(x)
    \]
    \begin{itemize}
        \item $h_o$ 是原始的输出。
        \item $x$ 是输入。
        \item $W$ 是权重,用于计算每个专家的激活值。
        \item $p$ 是一个可学习的参数,文中作者称之为“Domain-Aware Prompt”。
        \item $W_d$ 是门控权重。
        \item $\mathrm{TopK}$ 用于取激活值最高的专家参与运算。
    \end{itemize}
\end{frame}

\begin{frame}{均衡MOE负载}
    \begin{itemize}
        \item 在MOE中,我们存在负载不均衡的问题:某些专家大量被激活导致失去MOE的 sparse 特点,退化为 dense 结构。
    \end{itemize}
    \begin{itemize}
        \item \textbf{解决办法:Load-Balance Loss}
    \end{itemize}
    \[
        l_{aux} = \alpha \sum_{i=1}^{n} f_i \cdot p_i
    \]
    \begin{itemize}
        \item $f_i$ 是第 $i$ 个专家处理的 token 比例。
        \item $p_i$ 是第 $i$ 个专家的平均参数。
    \end{itemize}
\end{frame}
\begin{frame}{UP-Person: Unified Parameter-Efficient Transfer Learning for Text-based Person Retrieval}
    \framesubtitle{发表于: CAST 2025}

    \begin{columns}[T]
        % 左栏文字
        \begin{column}{0.5\textwidth}
            \textbf{研究动机 (Motivation):}
            \vspace{0.5cm}
            \begin{itemize}
                \item 引入Lora到CLIP微调中
                \item 对attention的K、V进行prefix填充
                \item 改进Adapter以均衡分布
            \end{itemize}
            
        \end{column}
        
        % 右栏图片
        \begin{column}{0.5\textwidth}
            \begin{figure}
                \centering
                \autoslidegraphic{figures/Lora.png}
                \caption{论文提出的模型架构图}
            \end{figure}
        \end{column}
    \end{columns}
\end{frame}

\begin{frame}{Lora对Attention权重进行微调}
    \begin{columns}[T]
        % 左栏文字
        \begin{column}{0.5\textwidth}
            \vspace{0.5cm}
            \begin{itemize}
                \item \textbf{Lora原理}
                \[
                    W_k = W_k + \alpha \cdot W_{\text{loraA}} W_{\text{loraB}}
                \]
                \item 其中 $W_k$ 是用于计算 $K$ 的 projection 矩阵
                \item 由于 $W_{\text{loraA}}$ 和 $W_{\text{loraB}}$ 的秩远远小于 $W_k$,我们可以实现高效的微调
            \end{itemize}
            
        \end{column}
        
        % 右栏图片
        \begin{column}{0.5\textwidth}
            \begin{figure}
                \centering
                \autoslidegraphic{figures/lora_math.png}
                \caption{Lora原理}
            \end{figure}
        \end{column}
    \end{columns}
\end{frame}

\begin{frame}{改进Adapter}
    \begin{columns}[T]
        % 左栏文字
        \begin{column}{0.5\textwidth}
            \textbf{研究动机 (Motivation):}
            \vspace{0.5cm}
            \begin{itemize}
                \item 之前:并行 Adapter: \[output = x + MLP(x) + Adapter(x)\] 
                \item 现在:合并到输入\[output = x + MLP(x + Adapter(x))\] 
            \end{itemize}
            
            \begin{itemize}
                \item \textbf{为什么要这么做}
                \item 作者认为传统 Adapter 位于 Attention 层容易与 LoRA/Prefix 产生结构重叠,将其移至 LayerNorm 层实现了空间解耦,有效消除了组件间的优化干扰。
            \end{itemize}
        \end{column}
        
        % 右栏图片
        \begin{column}{0.5\textwidth}
            \begin{figure}
                \centering
                \autoslidegraphic{figures/L-adapter.png}
                \caption{论文提出的模型架构图}
            \end{figure}
        \end{column}
    \end{columns}
\end{frame}

\begin{frame}{S-Prefix}
    \begin{columns}[T]
        % 左栏文字
        \begin{column}{0.5\textwidth}
            \textbf{研究动机 (Motivation):}
            \vspace{0.5cm}
            \begin{itemize}
                \item 在文本和视觉Transformer的每一层的Key和Value中引入可学习的前缀Token。
                \item K = [$P_k$ : K] $P_K$是prefix,K是原始的Key,二者拼接。
                \item V = [$P_v$ : V] $P_V$是prefix,V是原始的Value,二者拼接。
                \item 我个人的理解是注入行人检索特定的先验知识
            \end{itemize}
            
        \end{column}
        
        % 右栏图片
        \begin{column}{0.5\textwidth}
            \begin{figure}
                \centering
                \autoslidegraphic{figures/s_prefix.png}
                \caption{论文提出的模型架构图}
            \end{figure}
        \end{column}
    \end{columns}
\end{frame}
% ==================================================
% Section 3: IRRA
% ==================================================
\section{IRRA}
\begin{frame}{Cross-Modal Implicit Relation Reasoning and Aligning for Text-to-Image Person Retrieval}
    \framesubtitle{发表于: CVPR 2023}

    \begin{columns}[T]
        % 左栏文字
        \begin{column}{0.5\textwidth}
            \textbf{核心贡献:}
            \begin{itemize}
                \item 引入了cross attention,实现图像区域和文本token的细粒度交互
                \item 引入了Masked Langua Model(MLM),随机MASK token后做预测,提升模型的理解能力
                \item 引入了Similarity Distribution Match(SDM),改进了InfoNCE loss
            \end{itemize}
            
        \end{column}
        
        % 右栏图片
        \begin{column}{0.5\textwidth}
            \begin{figure}
                \centering
                \autoslidegraphic{figures/IRRA.png}
                \caption{论文提出的模型架构图}
            \end{figure}
        \end{column}
    \end{columns}
\end{frame}

\begin{frame}
    \frametitle{Cross Attention}
    \begin{columns}[T]
        % 左栏文字
        \begin{column}{0.5\textwidth}
            \vspace{0.5cm}
            \begin{itemize}
                \item 把经过视觉encoder的图像区域特征 $f_{vision}$ 和文本encoder的文本Token特征 $f_{text}$ 进行cross attention计算
                \item 文本特征作为Query,图像区域特征作为Key和Value
                \item 计算得到的注意力特征 $f_{mask}$ 送入到下面会提到的Masked Language Model 
            \end{itemize}
            
        \end{column}
        
        % 右栏图片
        \begin{column}{0.5\textwidth}
            \begin{figure}
                \centering
                \autoslidegraphic{figures/cross_atten.png}
                \caption{Cross Attention}
            \end{figure}
        \end{column}
    \end{columns}
\end{frame}

\begin{frame}
    \frametitle{Masked Language Model}
    \begin{columns}[T]
        % 左栏文字
        \begin{column}{0.5\textwidth}
            \vspace{0.5cm}
                        \begin{itemize}
                                \item 我们会以15\%的概率随机MASK掉文本Token,然后利用cross attention计算得到的 $f_{mask}$ 来预测被MASK掉的Token
                                \item 这样可以提升模型对文本的理解能力,理解图像的含义,作者称之为 Implicit Relation Reasoning (IRR)
                                \item 作者提出 IRR loss:\[
\mathcal{L}_{irr}
= - \frac{1}{|\mathcal{M}|\,|\mathcal{V}|}
    \sum_{i \in \mathcal{M}}
    \sum_{j \in |\mathcal{V}|}
    y_j^i \log
    \frac{\exp(m_j^i)}{\sum_{k=1}^{|\mathcal{V}|} \exp(m_k^i)} \, .
\] 
                        \end{itemize}
            
        \end{column}
        
        % 右栏图片
        \begin{column}{0.5\textwidth}
            \begin{figure}
                \centering
                \autoslidegraphic{figures/cross_atten.png}
                \caption{Cross Attention}
            \end{figure}
        \end{column}
    \end{columns}
\end{frame}


\begin{frame}
    \frametitle{Similarituy Distribution Matching}
    \begin{columns}[T]
        % 左栏文字
        \begin{column}{0.5\textwidth}
            \vspace{0.5cm}
            \begin{itemize}
                \item 传统的 InfoNCE loss 只关注正样本和负样本。在CLIP预训练中图文是一一对应的,但行人检索里一个图像对应多个文本描述,InfoNCE loss 无法刻画这种一对多关系。
                \item 我们把相似度转换为概率分布,用 KL 散度拉近模型给出的分布与真实分布:
\[
KL(\mathbf{p_i} \parallel \mathbf{q_i}) = \frac{1}{N} \sum_{i=1}^{N} \sum_{j=1}^{N} p_{i,j} \log\left(\frac{p_{i,j}}{q_{i,j} + \epsilon}\right)
\]
            \end{itemize}
            
        \end{column}
        
        % 右栏图片
        \begin{column}{0.5\textwidth}
            \begin{figure}
                \centering
                \autoslidegraphic{figures/SDM.png}
                \caption{Cross Attention}
            \end{figure}
        \end{column}
    \end{columns}
\end{frame}

\section{UFineBench}
\begin{frame}{UFineBench: Towards Text-based Person Retrieval with Ultra-fine Granularity}
    \framesubtitle{发表于: CVPR 2024}
    \begin{columns}[T]
        % 左栏文字
        \begin{column}{0.5\textwidth}
            \textbf{核心贡献:}
            \begin{itemize}
                \item 提出了一个全新的行人检索训练集,提供更加细粒度、描述更加丰富的文本描述
                \item 提出了一个综合性的评测基准,首次使用LLM来对数据集描述做改写增强
                \item 提出了一种新的评估指标,优于mAP 、 rank@k
                \item 提出一种基于decoder的细粒度对齐方法
            \end{itemize}
        \end{column}
        
        % 右栏图片
        \begin{column}{0.5\textwidth}
            \begin{figure}
                \centering
                \autoslidegraphic{figures/UFineBench.png}
                \caption{UFineBench 框架图}
            \end{figure}
        \end{column}
    \end{columns}
\end{frame}

\begin{frame}{UFine6926}
   \frametitle{新的数据集:UFine6926}
    \begin{columns}[T]
        % 左栏文字
        \begin{column}{0.5\textwidth}
            \vspace{0.5cm}
            \begin{itemize}
                \item 人工标注的全新数据集,平均每段描述用了80个词,远高于CUHK-PEDS(27个)
                \item 使用对比实验证明了在这个数据集上训练的模型rank@k, mAP等指标均有显著提升
            \end{itemize}
            
        \end{column}
        
        % 右栏图片
        \begin{column}{0.5\textwidth}
            \begin{figure}
                \centering
                \autoslidegraphic{figures/UFine6926.png}
                \caption{UFine6926 数据集效果}
            \end{figure}
        \end{column}
    \end{columns}
\end{frame}

\begin{frame}{UFine3C}
   \frametitle{新的验证集:UFine3C}
    \begin{columns}[T]
        % 左栏文字
        \begin{column}{0.5\textwidth}
            \vspace{0.5cm}
            \begin{itemize}
                \item 提供了不同分辨率、亮度、背景、场景的人物图片
                \item 借助LLM(Qwen llama)来对描述进行改写,提供不同风格,不同粒度的文本描述
            \end{itemize}
            
        \end{column}
        
        % 右栏图片
        \begin{column}{0.5\textwidth}
            \begin{figure}
                \centering
                \autoslidegraphic{figures/UFine3C.png}
                \caption{UFine6926 数据集效果}
            \end{figure}
        \end{column}
    \end{columns}
\end{frame}

\begin{frame}{}
   \frametitle{新的评价指标:mSD(mean similarity distribution)}
    \begin{columns}[T]
        % 左栏文字
        \begin{column}{0.5\textwidth}
            \vspace{0.5cm}
            \begin{itemize}
                \item 右图所示的情况,三组相似度不同,但是rank@k,mAP得出的结果不同
                \item 说明mAP与rank@k不能很好的捕捉到正负样本之间的相对值大小带来的差异
                \item 即[100(正) 99(负)] 和 [100(正) 0(负)]显然后者是更好的相似度
            \end{itemize}
        \end{column}
        
        % 右栏图片
        \begin{column}{0.5\textwidth}
            \begin{figure}
                \centering
                \autoslidegraphic{figures/MSD.png}
                \caption{UFine6926 数据集效果}
            \end{figure}
        \end{column}
    \end{columns}
\end{frame}

\begin{frame}{}
   \frametitle{使用decoder实现的细粒度对齐}
    \begin{columns}[T]
        % 左栏文字
        \begin{column}{0.5\textwidth}
            \vspace{0.5cm}
            \begin{itemize}
                \item 人们常常使用「CLS」token放在图片patch的首部,取这个token的hidden来作为我们的图像信息
                \item 相似的:    「EOS」token放在文本末尾,作为文本信息的总结。
                \item 但是在计算相似度的过程中,其余的token的信息完全不会被用到
            \end{itemize}
        \end{column}
        
        % 右栏图片
    \end{columns}
\end{frame}

\begin{frame}{}
   \frametitle{使用decoder实现的细粒度对齐}
    \begin{columns}[T]
        % 左栏文字
        \begin{column}{0.5\textwidth}
            \vspace{0.5cm}
            \begin{itemize}
                \item 训练公共的decoder和一个公共的 Query向量,来做完整的图文embedding对齐
                \item 即:把图文的完整embedding,分别和这个公共Query送入decoder,得倒新的图、文特征表达
                \item 用这个新的图、文特征来做对齐
            \end{itemize}
        \end{column}
        
        % 右栏图片
        \begin{column}{0.5\textwidth}
            \begin{figure}
                \centering
                \autoslidegraphic{figures/UFine_decoder.png}
                \caption{UFine6926 数据集效果}
            \end{figure}
        \end{column}
    \end{columns}
\end{frame}


% \section{方法与改进}
% \begin{frame}[fragile]{方法改进:跨模态注意力机制}
%     % [fragile] 是代码块必须的选项
    
%     \textbf{1. 数学定义:}
%     为了增强图像区域 $\vx$ 和文本 Token $\vt$ 的交互,计算如下相似度:
%    \[        
% %    Attention(Q, K, V) = \softmax\left(\frac{Q K^T}{\sqrt{d_k}}\right) V
%     \] 
    
%     \textbf{2. PyTorch 实现片段:}
%     \begin{lstlisting}
% # 定义 Cross-Attention 层
% class CrossAttention(nn.Module):
%     def forward(self, img_feat, text_feat):
%         # img_feat: [Batch, Img_Seq, Dim]
%         # text_feat: [Batch, Txt_Seq, Dim]
        
%         q = self.query(text_feat)
%         k = self.key(img_feat)
        
%         attn_output, _ = self.mha(q, k, k)
%         return attn_output
%     \end{lstlisting}
% \end{frame}

% % ==================================================
% % Section 4: 实验结果
% % ==================================================
% \section{实验结果}
% \begin{frame}{实验结果:定性分析 (Qualitative)}
%     \textbf{测试 Prompt:} \textit{“一只戴着墨镜的赛博朋克风格猫咪”}
    
%     \vspace{0.3cm}
    
%     \begin{figure}
%         \centering
%         % 第一张图
%         \begin{subfigure}{0.3\textwidth}
%             \centering
%             \rule{\linewidth}{2.5cm} % 替换为图片
%             \caption{Baseline}
%         \end{subfigure}%
%         \hspace{0.05\textwidth}
%         % 第二张图
%         \begin{subfigure}{0.3\textwidth}
%             \centering
%             \rule{\linewidth}{2.5cm} % 替换为图片
%             \caption{Ours (改进版)}
%         \end{subfigure}%
%         \hspace{0.05\textwidth}
%         % 第三张图
%         \begin{subfigure}{0.3\textwidth}
%             \centering
%             \rule{\linewidth}{2.5cm} % 替换为图片
%             \caption{Ground Truth}
%         \end{subfigure}
%     \end{figure}
    
%     \textbf{分析:} 可以看到我们的方法在光影细节(墨镜反射)上比 Baseline 更真实。
% \end{frame}

% ==================================================
% Section 5: 问题与计划
% ==================================================
\section{计划}
\begin{frame}{下周计划}
    \begin{columns}[T]
        
        
        \begin{column}{0.48\textwidth}
            \setbeamercolor{block title}{bg=blue!80!black,fg=white}
            \begin{block}{下周计划 (Next Steps)}
                \begin{enumerate}
                    \item 继续阅读行人检索领域的论文
                    \item 阅读这周所看部分论文作者开源的代码,尝试理解其中的代码实现
                \end{enumerate}
            \end{block}
        \end{column}
    \end{columns}
\end{frame}

% --- 结束页 ---
\begin{frame}
    \centering
    \Huge \textbf{Q \& A}
    
    \vspace{1cm}
    \large 感谢聆听,请老师指导!
\end{frame}

\end{document}